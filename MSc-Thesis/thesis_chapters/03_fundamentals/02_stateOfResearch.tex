

\section{State of research} \label{cha:stateOfResearch}

\subsection{Known security vulnerabilities in the use of Kubernetes}

Kubernetes has become the backbone of modern cloud-native application development, offering unparalleled scalability, flexibility, and efficiency in managing containerized workloads. However, its widespread adoption has also exposed significant security vulnerabilities, which, if left unaddressed, can lead to severe consequences for organizations. The complexity of Kubernetes, combined with its extensive configurability, has made it particularly susceptible to misconfigurations, privilege escalation, and distributed denial-of-service (DDoS) attacks. These vulnerabilities are not only technical challenges but also reflect gaps in the understanding and implementation of security best practices among Kubernetes users \citep{rahman2023, bose2021}.

One of the most critical security issues in Kubernetes environments is the prevalence of misconfigurations in Kubernetes manifests. Kubernetes manifests are declarative configuration files that define the desired state of the system, including the deployment of applications, resource allocation, and access controls. While these manifests provide a powerful mechanism for managing complex systems, they are also prone to errors. Misconfigurations, such as granting excessive permissions, using insecure default settings, or failing to enforce access controls, are among the most common sources of vulnerabilities. Rahman et al. (\citeyear{rahman2023}) conducted an empirical study analyzing over 2,000 Kubernetes manifests from open-source repositories and found that misconfigurations were widespread. Their findings revealed that many developers lack the necessary expertise to configure Kubernetes securely, leading to a high frequency of security issues. Similarly, Bose et al. (\citeyear{bose2021}) identified under-reported security defects in Kubernetes manifests through a qualitative analysis of over 5,000 commits from open-source repositories. Their study emphasized the need for systematic approaches to detect and address these vulnerabilities, as even minor misconfigurations can have significant security implications.

The risks associated with misconfigurations are further exacerbated by the dynamic and distributed nature of Kubernetes environments. For example, excessive permissions granted to service accounts or users can enable attackers to escalate their privileges and gain unauthorized access to sensitive resources. Insecure default settings, such as allowing unrestricted access to the Kubernetes API server, can expose the cluster to external threats. The lack of proper access controls, such as role-based access control (RBAC) policies, can also create opportunities for attackers to exploit vulnerabilities and compromise the system. These issues highlight the importance of implementing security best practices and leveraging automated tools to detect and remediate misconfigurations \citep{rahman2023, bose2021}.

In addition to misconfigurations, Kubernetes is vulnerable to a variety of attack vectors, including distributed denial-of-service (DDoS) attacks and privilege escalation. DDoS attacks, in particular, pose a significant threat to Kubernetes clusters, as they can exploit the platform's auto-scaling mechanisms to exhaust resources. Lee et al. (\citeyear{lee2023}) investigated the vulnerability of Kubernetes to YoYo attacks, a type of DDoS attack that repeatedly triggers the auto-scaling of resources, causing resource exhaustion and service degradation. Their study demonstrated that while Kubernetes exhibits better resilience to such attacks compared to traditional virtual machines, it remains vulnerable to resource exhaustion under sustained attack. The dynamic nature of Kubernetes, which is one of its greatest strengths, can also become a liability in the context of DDoS attacks, as the auto-scaling mechanism can inadvertently amplify the impact of the attack. To mitigate this risk, Lee et al. proposed a machine learning-based detection mechanism using the XGBoost classifier, which significantly improves the accuracy of attack detection. This approach highlights the potential of artificial intelligence and machine learning in enhancing the security of Kubernetes environments.

Privilege escalation is another critical security concern in Kubernetes. Attackers can exploit misconfigured RBAC policies, vulnerabilities in container runtimes, or weaknesses in DevOps pipelines to gain unauthorized access to sensitive resources. Once inside the cluster, attackers can move laterally, compromise additional resources, and exfiltrate data. Pecka et al. (\citeyear{pecka2022}) highlighted the risks associated with privilege escalation in Kubernetes environments, particularly in the context of DevOps pipelines. Their study emphasized the need for robust access controls and continuous monitoring to prevent unauthorized access and detect suspicious activity. The complexity of Kubernetes' access control mechanisms often makes it challenging for administrators to enforce the principle of least privilege, further exacerbating the risk of privilege escalation. This underscores the importance of adopting a comprehensive approach to security that includes both technical measures and organizational best practices \citep{pecka2022}.

To address these vulnerabilities, researchers and practitioners have proposed a range of detection and mitigation strategies. Automated tools, such as Dockersec and Lic-Sec, have been developed to enhance access control and profile generation in Kubernetes environments. Alyas et al. (\citeyear{alyas2022}) proposed Dockersec as an open-source tool for implementing mandatory access control (MAC) policies in Linux containers. Dockersec simplifies the process of defining and enforcing access control policies, making it easier for administrators to secure their Kubernetes environments. Similarly, Zhu et al. (\citeyear{zhu2023}) developed Lic-Sec, a tool for automatically generating security profiles based on the behavior of containers. Lic-Sec addresses the limitations of manual approaches to access control, which are often error-prone and time-consuming. By leveraging tools like Dockersec and Lic-Sec, organizations can improve the security of their Kubernetes clusters and reduce the risk of misconfigurations and other vulnerabilities.

Despite these advancements, the adoption of automated security tools in Kubernetes environments remains limited. One of the main barriers to adoption is the lack of awareness and expertise among Kubernetes users. Many organizations are still in the early stages of their Kubernetes journey and may not have the resources or knowledge to implement advanced security measures. This highlights the need for education and training to improve the security awareness of Kubernetes users and enable them to leverage the full potential of automated tools. Furthermore, the effectiveness of these tools depends on their integration into a broader security framework that includes regular audits, incident response plans, and continuous monitoring. By adopting a defense-in-depth approach to security, organizations can mitigate the risks associated with Kubernetes and ensure the reliability and resilience of their applications \citep{alyas2022, zhu2023}.

In conclusion, Kubernetes has revolutionized the way organizations deploy and manage applications, but its complexity and configurability have introduced significant security challenges. Misconfigurations, DDoS attacks, and privilege escalation are among the most pressing issues facing Kubernetes users today. Automated tools, such as Dockersec and Lic-Sec, offer promising solutions for detecting and mitigating these vulnerabilities, but their adoption remains limited. To address these challenges, organizations must adopt a comprehensive approach to security, combining technical measures, automated tools, and best practices. By doing so, they can mitigate the risks associated with Kubernetes and ensure the security and reliability of their applications.

\subsection{The field of use of SOAR tools}

Security Orchestration, Automation, and Response (SOAR) tools have emerged as a transformative technology in the field of IT security, offering significant potential to address the growing complexity and volume of cyber threats. Initially designed to enhance the efficiency of Security Operations Centers (SOCs), SOAR tools have since expanded their scope of application to include a variety of domains, such as Internet of Things (IoT) environments and cloud-native infrastructures. By automating routine tasks, orchestrating disparate security tools, and providing structured workflows, SOAR tools aim to streamline incident response and reduce the burden on SOC operators. Despite their promise, the adoption and evaluation of SOAR tools remain limited, with significant research gaps and challenges that need to be addressed to fully realize their potential \citep{bridges2023, empl2022}.

One of the primary applications of SOAR tools is in SOCs, where they are used to combat the pervasive issue of alert fatigue. SOC analysts are often overwhelmed by the sheer volume of alerts generated by traditional security systems, such as Security Information and Event Management (SIEM) tools. Many of these alerts are false positives, which can lead to wasted time and resources, as well as the risk of genuine threats being overlooked. SOAR tools address this challenge by automating the triage and prioritization of alerts, enabling analysts to focus on the most critical incidents. Additionally, SOAR tools provide structured workflows, known as playbooks, which standardize the incident response process and ensure that best practices are followed. These capabilities not only improve the efficiency of SOC operations but also enhance the overall security posture of organizations \citep{bridges2023}.

Beyond traditional SOCs, SOAR tools are increasingly being applied in IoT environments, where they play a critical role in addressing the unique security challenges posed by connected devices. The proliferation of IoT devices in both consumer and industrial settings has created new attack surfaces for cybercriminals, who can exploit vulnerabilities in these devices to gain unauthorized access to networks and data. Empl et al. (\citeyear{empl2022}) explored the potential of SOAR tools to enhance IoT security and proposed a framework for their integration using Digital Twins. Digital Twins are virtual representations of physical assets that mirror their state and behavior in real time. By leveraging this technology, SOAR tools can monitor IoT devices for anomalies, automate the detection and mitigation of threats, and provide actionable insights to security teams. This approach highlights the versatility of SOAR tools and their ability to adapt to emerging security challenges in diverse environments.

Despite their growing adoption, the evaluation and implementation of SOAR tools face several challenges, which have limited the availability of empirical studies in this field. One of the main barriers is the difficulty of creating realistic test environments that accurately replicate the conditions of a SOC. SOAR tools are designed to integrate with a wide range of security systems and data sources, which makes their evaluation complex and resource-intensive. Bridges et al. (\citeyear{bridges2023}) conducted an empirical study comparing six commercial SOAR tools and identified significant gaps in the standardization of evaluation frameworks. Their findings revealed that the lack of standardized metrics and methodologies for assessing the performance and effectiveness of SOAR tools hinders their adoption and limits the ability of organizations to make informed decisions about their implementation. This underscores the need for further research to develop comprehensive evaluation frameworks that can guide the selection and deployment of SOAR tools in different contexts.

Another challenge is the integration of SOAR tools with existing security systems and workflows. Many organizations struggle to align the capabilities of SOAR tools with their specific security requirements and operational processes. This is particularly true in cloud-native environments, where the dynamic and distributed nature of applications creates unique security challenges. Kubernetes, for example, has become the de facto standard for container orchestration, but its complexity and configurability have made it a prime target for cyberattacks. The integration of SOAR tools with Kubernetes offers a promising solution to these challenges, as it can automate threat detection and response, enforce security best practices, and provide a centralized platform for managing security operations. Empl et al. (\citeyear{empl2022}) highlighted the potential of SOAR tools to address security challenges in Kubernetes environments and called for further research to explore their application in this domain.

The future of SOAR tools lies in their ability to adapt to the evolving threat landscape and integrate with emerging technologies. As cyber threats become more sophisticated, the need for advanced security solutions that leverage artificial intelligence (AI) and machine learning (ML) will continue to grow. SOAR tools are well-positioned to incorporate these technologies, enabling them to detect and respond to threats more effectively. For example, AI-powered analytics can enhance the ability of SOAR tools to identify patterns and anomalies in security data, while ML algorithms can improve the accuracy of threat detection and prediction. Additionally, the development of open-source SOAR tools and frameworks can help address the barriers to adoption by providing organizations with cost-effective and customizable solutions \citep{bridges2023, empl2022}.

In conclusion, SOAR tools represent a significant advancement in IT security, with applications that extend beyond traditional SOCs to include IoT environments and cloud-native infrastructures. Their ability to automate routine tasks, orchestrate security operations, and provide structured workflows makes them a valuable asset for organizations seeking to enhance their security posture. However, the adoption and evaluation of SOAR tools remain limited, with significant research gaps and challenges that need to be addressed. By developing standardized evaluation frameworks, improving integration capabilities, and leveraging emerging technologies, SOAR tools can continue to evolve and play a critical role in addressing the security challenges of the future.

\section{Summary}

The research focuses on the integration of Security Orchestration, Automation, and Response (SOAR) tools into Kubernetes clusters, analyzing their impact on performance and security. Kubernetes, as a leading platform for managing containerized applications, has become a cornerstone of modern cloud-native development. However, its complexity and configurability introduce significant security challenges, such as misconfigurations, privilege escalation, and distributed denial-of-service (DDoS) attacks. Real-world incidents, like the Tesla AWS breach and the Capital One data breach, highlight the severe consequences of these vulnerabilities. Automated tools, such as SLI-KUBE and Dockersec, and machine learning-based solutions have been proposed to mitigate these risks, but their adoption remains limited.

SOAR tools, introduced as an evolution of SIEM systems, aim to enhance the efficiency of Security Operations Centers (SOCs) by automating routine tasks, orchestrating disparate tools, and providing structured workflows. Their application extends beyond SOCs to IoT environments and cloud-native infrastructures, including Kubernetes. Despite their potential, SOAR tools face challenges in adoption, including integration difficulties and a lack of standardized evaluation frameworks. Research highlights the need for further studies to assess their effectiveness in real-world scenarios.

The study aims to explore the trade-offs between performance and security when implementing SOAR tools in Kubernetes clusters. By addressing these challenges, the research seeks to provide actionable insights for organizations, enabling them to enhance security without compromising performance, and to contribute to the development of best practices for integrating SOAR tools into cloud-native environments.
