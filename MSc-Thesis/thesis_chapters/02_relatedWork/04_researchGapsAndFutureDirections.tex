

\chapter{Research Gaps and Future Directions} \label{cha:researchGapsAndFutureDirections}

While significant progress has been made in understanding Kubernetes security and the potential of SOAR tools, several research gaps remain. First, there is a lack of empirical studies evaluating the impact of SOAR tools on Kubernetes performance and security. Most existing research focuses on theoretical frameworks or isolated case studies, leaving a gap in practical, real-world evaluations.

Second, the scalability of SOAR tools in large-scale Kubernetes deployments is an area that requires further investigation. As Kubernetes clusters grow in size and complexity, the performance of SOAR tools may be affected, necessitating the development of optimized algorithms and architectures.

Finally, the integration of machine learning techniques into SOAR tools for Kubernetes security is a promising avenue for future research. Lee et al. (\citeyear{lee_experimental_2023}) and Rahman et al. (\citeyear{rahman_security_2023}) demonstrate the potential of machine learning for detecting and mitigating security threats, but more work is needed to integrate these capabilities into SOAR tools.
