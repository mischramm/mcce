

\section{Problem stating} \label{cha:problem}

%Zitate richtig einfügen
% Text über Kubernetes umschreiben um ihn allgemeiner zu halten, sonst wiederholt sich vieles mit Kapitel 2.1.1

Kubernetes has become the defining tool since its introduction in 2014. Companies such as Adidas, IBM and even the US Department of Defense rely on this open source software solution (\cite{rahman_security_2023}). Kubernetes, also abbreviated as K8s, has its origins in Google's Borg, which has been used for over a decade to manage containerized workloads in Google's internal clusters. Kubernetes aims to enable the automation of deployment, management and scaling of containerized applications through efficient management of heterogeneous resources, improving and facilitating the management of cloud-native applications (\cite{deng_cloud-native_2023}). Practitioners use Kubernetes because it reduces repetitive manual processes in the deployment and management of containers, significantly speeding up software deployment or load times, for example (\cite{bose_under-reported_2021, islam_shamim_xi_2020, rahman_security_2023})(Bose et al., 2021; Islam Shamim et al., 2020; Rahman et al., 2023).

The CNCF survey found that 96\% of respondents either already use or are considering using Kubernetes in their organizations, and 69\% are already using it productively. 5.6 million developers have used Kubernetes at this point in time, which is a significant increase compared to the previous year (CNCF - Annual Survey 2021, 2021). Kubernetes is omnipresent and, together with cloud-native applications, continues to gain ground. One disadvantage of Kubernetes is therefore IT security. In another survey, 44 percent of users surveyed stated that they had postponed the deployment of individual applications due to security concerns and as many as 95 percent reported having been affected by an incident (\cite{shamim_mitigating_2021}). Attackers gained unauthorized access to Tesla's AWS via an unsecured Kubernetes dashboard in 2018 (\cite{islam_shamim_xi_2020}).

The threat situation is becoming increasingly intense for users. The COVID-19 pandemic in particular has led to an internationalization of cybercrime, especially through malware. According to the Kaspersky Security Bulletin, 400,000 malware files were in circulation every day in 2022, which corresponds to an increase of 5\% compared to the previous year (\cite{djenna_artificial_2023}). Drivers behind this development include the concept of mobile or remote working in environments with poor IT security, which has become widespread since the pandemic (\cite{ferreira_recommender_2023}). However, more and more Internet of Things applications are also leading to more and more smart devices being connected to the Internet in both the commercial and private sectors, but this also creates serious security gaps that can be used as gateways by cyber criminals (\cite{empl_soar4iot_2022}).

Due to new generation attacks and evasion techniques, traditional protection systems such as firewalls, intrusion detection systems, antivirus software, access control lists, etc. are no longer effective in detecting these sophisticated attacks. Therefore, there is an urgent need to find innovative and more viable solutions to prevent cyberattacks (\cite{aslan_comprehensive_2023}). Artificial intelligence is definitely a focus topic for IT security experts. Machine learning (ML) and deep learning (DL) have advanced IT security and can make a contribution when it comes to detecting, classifying and analyzing and thus mitigating cyberattacks (\cite{djenna_unmasking_2023}).

One of these innovative new tools is SOAR - Security Orchestration, Automation and Response - which has been under discussion since 2017. This is the latest generation of IT security tools designed to enable operators of SOCs - Security Operations Centers - to combat threats efficiently and process them in a uniform or automated manner. SOAR is therefore also a further evolution of SIEM - Security Information and Event Management (\cite{empl_soar4iot_2022}). Orchestration refers to the ability to bring together different tools and sources in one system to simplify and speed up the investigation of an incident. Automation means reducing the manual effort required by SOC operators (\cite{katsikas_computer_2022}). Configurable workflows and playbooks are used to standardize and accelerate incident response. This means that analysts are not faced with a huge, unstructured pool of data, but are given context and structure (\cite{bridges_testing_2023}).

So while Kubernetes has become the absolute top dog, SOAR tools have hardly been researched. This is due to the novelty of these tools and also because there are barriers to empirical investigation and test environments, as SOAR must be integrated into SOCs (\cite{bridges_testing_2023}). At the same time, empirical studies are urgently needed as more and more manufacturers are entering the market with SOAR products and propagating their effectiveness (\cite{empl_soar4iot_2022}).
